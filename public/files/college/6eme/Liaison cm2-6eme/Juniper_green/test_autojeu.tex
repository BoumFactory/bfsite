
\begin{multicols}{2}
\boite{Matériel :}{
\begin{itemize}
    \item un plateau de jeu (annexe 1 : nombres de 1 � 20 ; annexe 2 : nombres de 1 � 40 ; annexe 3 : nombres de 1 � 100) ; \textit{on pourra photocopier l�une des annexes et la glisser dans une pochette transparente ou bien travailler directement sur des transparents} ;
    \item des feutres effa�ables.
\end{itemize}
}

\columnbreak

\boite{Règles du jeu}{
\begin{itemize}
    \item Le joueur qui commence barre un nombre pair.
    \item Ensuite, chaque joueur, � tour de r�le, barre un nombre parmi les multiples ou les diviseurs du nombre barr� par son adversaire.
\end{itemize}
}
\end{multicols}

\boite{Objectif : }{
    \textbf{Un joueur est d�clar� gagnant lorsque son adversaire ne peut plus jouer.}
}

\boite{Exemple de tour de jeu :}{
\begin{itemize}
    \item Si le joueur A barre le 12, le joueur B doit barrer un multiple ou un diviseur de 12. Il n�y a pas de multiple de 12 sur le plateau, il a donc le choix entre les diviseurs de 12 : 1 ; 2 ; 3 ; 4 et 6. Supposons que le joueur B barre le 3.
    \item Le joueur A peut alors barrer le 1, le 6, le 9, le 15 ou le 18. Supposons qu�il barre le 9.
    \item Le joueur B ne peut plus barrer que le 1 ou le 18. Supposons qu�il barre le 1.
    \item Le joueur A peut alors barrer n�importe quel nombre restant sur le plateau. Supposons qu�il barre le 19. Le joueur B ne peut plus jouer. Le joueur A est d�clar� gagnant.
\end{itemize}
}

\newpage

