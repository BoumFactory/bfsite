\begin{multicols}{2}
\boite{Matériel :}{
\begin{itemize}
    \item La feuille de jeu
    \item Un stylo rouge
    \item Un stylo bleu
\end{itemize}
}

\columnbreak

\boite{But :}{
Faire le plus possible d'alignements d'au moins trois nombres en ligne ou en colonne, contigus ou non, en les entourant de sa couleur.
}
\end{multicols}

\begin{multicols}{2}
\boite{Nombre de joueurs :}{
2 joueurs.
}

\columnbreak

\boite{Mécanisme :}{
Un coup est valide quand le joueur annonce un produit égal au nombre qu'il a choisi sur la grille. Les produits doivent se situer dans la limite de 10 x 10 pour le niveau 1 et de 12 x 12 pour le niveau 2.\\

Lien vers la démo : \href{https://jeux2maths.fr/multiplicato/}{https://jeux2maths.fr/multiplicato/}
}
\end{multicols}

\begin{multicols}{2}
\boite{Déroulement d'une partie :}{
\begin{itemize}
    \item Un joueur se munit d'un stylo rouge, l'autre d'un stylo bleu. La feuille de jeu est placée entre les joueurs.
    \item L'adversaire impose au joueur de choisir un nombre dans une table de multiplication pour laquelle il existe un produit disponible sur la grille.
    \item Le joueur annonce le produit correspondant au nombre qu'il choisit sur la grille, entoure ce nombre lorsque son adversaire a validé le choix et écrit le produit dans sa colonne sur la feuille de jeu. Il impose à son tour une table de multiplication dans laquelle son adversaire devra jouer.
    \item Si l'adversaire propose une table pour laquelle il n'existe plus de nombre disponible sur la grille, le joueur entoure un nombre de son choix.
    \item Si le joueur commet une erreur dans son produit, il ne peut jouer, et c'est alors l'adversaire qui entoure un nombre de son choix.
    \item La partie se termine quand tous les nombres sont entourés sur la grille.
\end{itemize}
}

\columnbreak

\boite{Scores :}{
Les points sont comptés à l'issue de la partie.
\begin{itemize}
    \item Un alignement de 3 nombres dans une couleur vaut 1 point.
    \item Un alignement de 4 nombres dans une couleur vaut 3 points.
    \item Un alignement de 5 nombres dans une couleur vaut 10 points.
\end{itemize}
Le vainqueur est le joueur qui totalise le plus de points à l'issue de la partie.

Remarque : on peut également comptabiliser les alignements effectués en diagonale, mais le comptage s'avère alors difficile pour la majeure partie des élèves.
}
\end{multicols}