
\begin{center}
(Pour 2 joueurs)
\end{center}

\noindent
D’après la revue « Pour la science n° de juillet 97 ». Ian Stewart y décrit le jeu développé à l’origine par Richard Porteous, enseignant à l’école de Juniper Green.

\section*{MATÉRIEL}
\begin{itemize}
    \item un plateau de jeu (annexe 1 : nombres de 1 à 20 ; annexe 2 : nombres de 1 à 40 ; annexe 3 : nombres de 1 à 100) ; \textit{on pourra photocopier l’une des annexes et la glisser dans une pochette transparente ou bien travailler directement sur des transparents} ;
    \item des feutres effaçables.
\end{itemize}

\section*{RÈGLES DU JEU}
\begin{itemize}
    \item Le joueur qui commence barre un nombre pair.
    \item Ensuite, chaque joueur, à tour de rôle, barre un nombre parmi les multiples ou les diviseurs du nombre barré par son adversaire.
\end{itemize}

\noindent
\framebox{
\begin{minipage}{0.9\textwidth}
    \textbf{Un joueur est déclaré gagnant lorsque son adversaire ne peut plus jouer.}
\end{minipage}
}

\section*{Exemple : Sur un plateau de 20 nombres}
\begin{itemize}
    \item Si le joueur A barre le 12, le joueur B doit barrer un multiple ou un diviseur de 12. Il n’y a pas de multiple de 12 sur le plateau, il a donc le choix entre les diviseurs de 12 : 1 ; 2 ; 3 ; 4 et 6. Supposons que le joueur B barre le 3.
    \item Le joueur A peut alors barrer le 1, le 6, le 9, le 15 ou le 18. Supposons qu’il barre le 9.
    \item Le joueur B ne peut plus barrer que le 1 ou le 18. Supposons qu’il barre le 1.
    \item Le joueur A peut alors barrer n’importe quel nombre restant sur le plateau. Supposons qu’il barre le 19. Le joueur B ne peut plus jouer. Le joueur A est déclaré gagnant.
\end{itemize}


\newpage

\section*{Annexe 1 : Plateau de jeu (nombres de 1 à 20)}

\begin{center}
\begin{tabular}{|c|c|c|c|c|}
\hline
1 & 2 & 3 & 4 \\
\hline
5 & 6 & 7 & 8 \\
\hline
9 & 10 & 11 & 12 \\
\hline
13 & 14 & 15 & 16 \\
\hline
17 & 18 & 19 & 20 \\
\hline
\end{tabular}
\end{center}

\newpage

\section*{Annexe 2 : Plateau de jeu (nombres de 1 à 40)}

\begin{center}
\begin{tabular}{|c|c|c|c|c|}
\hline
1 & 2 & 3 & 4 & 5 \\
\hline
6 & 7 & 8 & 9 & 10 \\
\hline
11 & 12 & 13 & 14 & 15 \\
\hline
16 & 17 & 18 & 19 & 20 \\
\hline
21 & 22 & 23 & 24 & 25 \\
\hline
26 & 27 & 28 & 29 & 30 \\
\hline
31 & 32 & 33 & 34 & 35 \\
\hline
36 & 37 & 38 & 39 & 40 \\
\hline
\end{tabular}
\end{center}

\newpage

\section*{Annexe 3 : Plateau de jeu (nombres de 1 à 100)}

\begin{center}
\begin{tabular}{|c|c|c|c|c|c|c|c|c|c|}
\hline
1 & 2 & 3 & 4 & 5 & 6 & 7 & 8 & 9 & 10 \\
\hline
11 & 12 & 13 & 14 & 15 & 16 & 17 & 18 & 19 & 20 \\
\hline
21 & 22 & 23 & 24 & 25 & 26 & 27 & 28 & 29 & 30 \\
\hline
31 & 32 & 33 & 34 & 35 & 36 & 37 & 38 & 39 & 40 \\
\hline
41 & 42 & 43 & 44 & 45 & 46 & 47 & 48 & 49 & 50 \\
\hline
51 & 52 & 53 & 54 & 55 & 56 & 57 & 58 & 59 & 60 \\
\hline
61 & 62 & 63 & 64 & 65 & 66 & 67 & 68 & 69 & 70 \\
\hline
71 & 72 & 73 & 74 & 75 & 76 & 77 & 78 & 79 & 80 \\
\hline
81 & 82 & 83 & 84 & 85 & 86 & 87 & 88 & 89 & 90 \\
\hline
91 & 92 & 93 & 94 & 95 & 96 & 97 & 98 & 99 & 100 \\
\hline
\end{tabular}
\end{center}

