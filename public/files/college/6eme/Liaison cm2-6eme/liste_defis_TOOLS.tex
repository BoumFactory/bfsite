\section{Liste des défis}

\def\points{1}
\def\rdifficulty{3}
\begin{EXO}{}{can6a-2022}

    

	\begin{enumerate}[itemsep=1em, label=\arabic*)]
		\item \itempoint{1}$7 \times 7=$ $\ldots$
		\item \itempoint{1}La moitié de $36$ est : \ldots
		\item \itempoint{1}Complète : \\$19+\ldots =100$ 
		\item \itempoint{1}$3$ cahiers coûtent $9$\,\euro{}.\\ 				 $9$ cahiers coûtent $\ldots$\,\euro{}
		\item \itempoint{1}$2$ h $30$ min $=$ \\ $\ldots$ min
		\item \itempoint{1}Quel est le nombre écrit sous le point d'interrogation ?\\\tikzinclude{gGnt}
		\item \itempoint{1}$32+19=$$\ldots$
		\item \itempoint{1}$18$ élèves se mettent par groupe de $3$. \\ 			  Il y a $\ldots$ groupes.
		\item \itempoint{1}Le tiers de $27$ est :  $\ldots$ 
		\item \itempoint{1}Complète :\\ 				$4+9=\ldots+5$
		\item \itempoint{1}$4{,}4\times 10=$$\ldots$
		\item \itempoint{1}Un film commence à $19$ h $35$ et se termine à $21$ h $15$.\\ 			  Combien de temps a duré le film ?
		\item \itempoint{1}Complète :\\$3=$$\ldots$ quarts
		\item \itempoint{1}Ajoute $25$ min à $7$ h $50$ min.
		\item \itempoint{1}Ajoute un dixième à $2{,}96$.
		\item \itempoint{1}Yann a $30$ billes. Il a $8$ billes de moins que Lou.\\ 				   Lou a $\ldots$ billes.
		\item \itempoint{1}$0{,}2$ kg  $=$   $\ldots$ g
		\item \itempoint{1}Écris en chiffres : \\ 				  Deux-millions-deux-mille 
		\item \itempoint{1}Complète : \\ 				$10$ jours $=$$\ldots$ h
		\item \itempoint{1}Complète : \\ 			  $9$ heures $=$$\ldots$ min
		\item \itempoint{1}Combien faut-il de pièces de $10$ centimes pour avoir $5{,}80$\,\euro{}. \\ 						
		\item \itempoint{1}$0{,}33+0{,}4=$$\ldots$ 
		\item \itempoint{1}Le double de $4{,}8$ est $\ldots$
		\item \itempoint{1}Compléter :\\ $ .... \times 6=36$
	\end{enumerate}
	


\exocorrection


    
\begin{enumerate}[itemsep=1em, label=\arabic*)]
\item $7 \times 7={\color[HTML]{f15929}\boldsymbol{49}}$
\item La moitié de $36$ est $36\div 2={\color[HTML]{f15929}\boldsymbol{18}}$.
\item $100-19={\color[HTML]{f15929}\boldsymbol{81}}$
\item $3$ cahiers coûtent $9$\,\euro{}.\\ 			$3\times3=9$ cahiers coûtent $3\times9={\color[HTML]{f15929}\boldsymbol{27}}$\,\euro{}.
\item $2$ h $30$ min $=2\times 60+ 30$ min $={\color[HTML]{f15929}\boldsymbol{150}}$ min
\item Le nombre écrit sous le point d'interrogation est : ${\color[HTML]{f15929}\boldsymbol{90}}$.
\item $32+19=32+20-1=52-1={\color[HTML]{f15929}\boldsymbol{51}}$
\item Le nombre de groupes est donné par $18\div 3={\color[HTML]{f15929}\boldsymbol{6}}$.
\item Le tiers de $27$ est : $27\div 3={\color[HTML]{f15929}\boldsymbol{9}}$.
\item Le nombre cherché est : $4+9-5={\color[HTML]{f15929}\boldsymbol{8}}$.
\item $4{,}4\times 10={\color[HTML]{f15929}\boldsymbol{44}}$ 
\item Pour aller à $20$ h, il faut $25$ min, et il faut ajouter $1$ heure et $15$ min pour arriver à $21$ h $15$, soit au total ${\color[HTML]{f15929}\boldsymbol{1}}$ h ${\color[HTML]{f15929}\boldsymbol{40}}$ min.
\item $3=\dfrac{12}{4}=12\times \dfrac{1}{4}$, donc ${\color[HTML]{f15929}\boldsymbol{12}}$ quarts $=3$. 
\item Pour aller à $8$ h, il faut $10$ min, et il reste $15$ min à ajouter, ce qui donne ${\color[HTML]{f15929}\boldsymbol{8}}$ h et ${\color[HTML]{f15929}\boldsymbol{15}}$ min.
\item $1$ dixième $=0,1$, d'où $2{,}96+0,1 ={\color[HTML]{f15929}\boldsymbol{3{,}06}}$
\item Yann a $8$ billes de moins que Lou, donc Lou en a $8$ de plus, soit $30+8={\color[HTML]{f15929}\boldsymbol{38}}$ billes.
\item  Comme $1$ kg $=1\,000$ g,  pour passer des "kg" au "g", on multiplie par $1\,000$.\\ 			Comme : $0{,}2\times 1\,000 =200$, alors $0{,}2$ kg$={\color[HTML]{f15929}\boldsymbol{200}}$ g.
\item Deux-millions-deux-mille-deux $=2\,000\,000  + 2\,000 + 2={\color[HTML]{f15929}\boldsymbol{2\,002\,000}}$. 
\item Dans une journée, il y a $24$ heures, donc dans $10$ jours, il y a $10\times 24={\color[HTML]{f15929}\boldsymbol{240}}$ heures.
\item Dans une heure, il y a $60$ minutes, donc dans $9$ heures, il y a $9\times 60={\color[HTML]{f15929}\boldsymbol{540}}$ minutes.
\item Il faut : $5{,}8\div 0,1=5{,}8\times 10={\color[HTML]{f15929}\boldsymbol{58}}$ pièces.
\item  $0{,}33+0{,}4={\color[HTML]{f15929}\boldsymbol{0{,}73}}$
\item Le double de $4{,}8$ est $2\times 4{,}8={\color[HTML]{f15929}\boldsymbol{9{,}6}}$.
\item $ {\color[HTML]{f15929}\boldsymbol{6}} \times 6=36$
\end{enumerate}



\end{EXO}
\def\points{1}
\def\rdifficulty{2}
\begin{EXO}{}{canc3a-2023}

    

\begin{enumerate}[itemsep=1em, label=\arabic*)]
	\item \itempoint{1}$9 \times 4$
	\item \itempoint{1}$36+29$
	\item \itempoint{1}Combien y a-t-il de boules noires ? \\ 
\tikzinclude{ocLB}

	\item \itempoint{1}La moitié de $42$
	\item \itempoint{1}Complète :  $\ldots \times \ldots =35$
	\item \itempoint{1}\Temps{;;;;35;}+ \Temps{;;;;40;}
	\item \itempoint{1}Pour partager $30$ oeufs, combien de boites de  $6$ oeufs dois-je utiliser ? 
	\item \itempoint{1}Écris en chiffres le nombre cinquante-deux-mille-sept.
	\item \itempoint{1}Karole a $12$ ans. \\         Laurent a 5 ans de moins que Karole. Laurent a $\ldots$ ans
	\item \itempoint{1}Donne l'écriture décimale de  $3\times 7$ centièmes.
	\item \itempoint{1}Complète : \,\,\,  $1{,}8+\ldots =10$ 
	\item \itempoint{1}Complète : \,\,\,  $405= \ldots$ dizaines  $\ldots$  unités
	\item \itempoint{1}$25\div 5$
	\item \itempoint{1}Si $2$ cahiers coûtent $8$\,\euro{}, alors $8$ cahiers coûtent  $\ldots$\,\euro{}.
	\item \itempoint{1}$92\times 5$
	\item \itempoint{1}Dans $32$ combien de fois $4$ ?
	\item \itempoint{1}Complète : \,\,\, $7$ centaines et  $\ldots$  dizaines font  $740$.
	\item \itempoint{1}Combien de dixièmes y a-t-il en tout dans $8{,}48$ ?
	\item \itempoint{1}$1{,}93+ 0{,}8$
\end{enumerate}



\exocorrection


    
\begin{enumerate}[itemsep=1em, label=\arabic*)]
\item $9 \times 4={\color[HTML]{f15929}\boldsymbol{36}}$
\item $36+29=36+30-1=66-1={\color[HTML]{f15929}\boldsymbol{65}}$
\item Le nombre de boules noires est donné par : $8\times 3={\color[HTML]{f15929}\boldsymbol{24}}$.
\item La moitié de $42$ est $42\div 2={\color[HTML]{f15929}\boldsymbol{21}}$.
\item Deux réponses possibles (avec des entiers) : \\         ${\color[HTML]{f15929}\boldsymbol{5}}\times {\color[HTML]{f15929}\boldsymbol{7}}=35$\\         ${\color[HTML]{f15929}\boldsymbol{1}}\times {\color[HTML]{f15929}\boldsymbol{35}}=35$ 
\item De $40 \text{ min }$ pour aller à $1$ h, il faut $20$ min, et il reste $15$ min à ajouter.\\         On obtient  ${\color[HTML]{f15929}\boldsymbol{1}}$ h et ${\color[HTML]{f15929}\boldsymbol{15}}$ min.
\item Le nombre de boites est donné par $30\div 6={\color[HTML]{f15929}\boldsymbol{5}}$.
\item cinquante-deux-mille-sept = $52\,000$ + 7 = ${\color[HTML]{f15929}\boldsymbol{52\,007}}$ 
\item Puisque Laurent a 5 ans de moins que Karole, son âge est  : $12-5={\color[HTML]{f15929}\boldsymbol{7}}$ {\color[HTML]{f15929}ans}. 
\item $1$ centième $=0,01$, d'où $3\times 7$ centièmes $=3\times 7\times 0,01={\color[HTML]{f15929}\boldsymbol{0.21}}$.
\item Le nombre cherché est donné par : $10-1{,}8={\color[HTML]{f15929}\boldsymbol{8{,}2}}$.
\item $405 = {\color[HTML]{f15929}\boldsymbol{40}}$ dizaines ${\color[HTML]{f15929}\boldsymbol{5}}$ unités
\item $25\div 5={\color[HTML]{f15929}\boldsymbol{5}}$
\item $2$ cahiers coûtent $8$\,\euro{}.\\ $4\times2=8$ cahiers coûtent $4\times8={\color[HTML]{f15929}\boldsymbol{32}}$\,\euro{}.
\item $92\times 5=92\times 10 \div 2=920\div 2={\color[HTML]{f15929}\boldsymbol{460}}$
\item Dans $32$, il y a ${\color[HTML]{f15929}\boldsymbol{8}}$ fois $4$ car $8\times 4=32$.
\item $740 = 7$ centaines et ${\color[HTML]{f15929}\boldsymbol{4}}$ dizaines
\item $8{,}48 = 8$ unités $4$ dixièmes $8$ centièmes.\\Or $1$ unité = $10$ dixièmes donc $8$ unités $= 80$ dixièmes.\\Finalement $8{,}48 = 84$ dixièmes $8$ centièmes.\\Il y a donc ${\color[HTML]{f15929}\boldsymbol{84}}$ dixièmes en tout dans $8{,}48$.
\item $1{,}93+ 0{,}8={\color[HTML]{f15929}\boldsymbol{2{,}73}}$
\end{enumerate}



\end{EXO}
\def\points{1}
\def\rdifficulty{1}
\begin{EXO}{}{canc3a}

    
\begin{multicols}{2}
\begin{enumerate}[itemsep=1em, label=\arabic*)]
	\item \itempoint{1}\begin{minipage}[t]{\linewidth} $22+13$ \end{minipage}
	\item \itempoint{1}\begin{minipage}[t]{\linewidth} $65-32$ \end{minipage}
	\item \itempoint{1}\begin{minipage}[t]{\linewidth} $17+25$ \end{minipage}
	\item \itempoint{1}\begin{minipage}[t]{\linewidth} $83-25$ \end{minipage}
	\item \itempoint{1}\begin{minipage}[t]{\linewidth} $3\times 1\,000 + 6\times 10 + 5\times 100$ \end{minipage}
	\item \itempoint{1}\begin{minipage}[t]{\linewidth} $2{,}2+3$ \end{minipage}
	\item \itempoint{1}\begin{minipage}[t]{\linewidth} $1{,}4+1{,}16$ \end{minipage}
	\item \itempoint{1}\begin{minipage}[t]{\linewidth} $5{,}63-2{,}2$ \end{minipage}
	\item \itempoint{1}\begin{minipage}[t]{\linewidth} $6{,}1-4{,}5$ \end{minipage}
	\item \itempoint{1}\begin{minipage}[t]{\linewidth} J'ai $18$ ans. Je suis $2$ fois plus âgé que Joachim.\\Quel âge a Joachim ? \end{minipage}
	\item \itempoint{1}\begin{minipage}[t]{\linewidth} Léa a $17$ ans. Sa sœur a $5$ ans.\\Quelle est leur différence d'âge ? \end{minipage}
	\item \itempoint{1}\begin{minipage}[t]{\linewidth} Joachim a couru $2$ séquences de $10$ minutes. Combien de minutes a-t-il couru en tout ? \end{minipage}
	\item \itempoint{1}\begin{minipage}[t]{\linewidth} $\ldots - 3=2{,}2$ \end{minipage}
	\item \itempoint{1}\begin{minipage}[t]{\linewidth} $5 \times 6$ \end{minipage}
	\item \itempoint{1}\begin{minipage}[t]{\linewidth} $6 \times 4$ \end{minipage}
	\item \itempoint{1}\begin{minipage}[t]{\linewidth} On a coupé $4{,}1$ cm d'une ficelle qui en faisait $7{,}2$.\\Combien de centimètres en reste-t-il ? \end{minipage}
	\item \itempoint{1}\begin{minipage}[t]{\linewidth} \tikzinclude{TIsR}\end{minipage}
	\item \itempoint{1}\begin{minipage}[t]{\linewidth} $8 \times 7$ \end{minipage}

	\item \itempoint{1}\begin{minipage}[t]{\linewidth} $\ldots \times 20=190$ \end{minipage}
	\item \itempoint{1}\begin{minipage}[t]{\linewidth} $3$ kg de fraises coûtent $22{,}5$\,\euro{}, combien coûtent $12$ kg de fraises ? \end{minipage}
	\item \itempoint{1}\begin{minipage}[t]{\linewidth} $\ldots \times 4=40$ \end{minipage}
	\item \itempoint{1}\begin{minipage}[t]{\linewidth} Le diamètre d'un cercle de $60$ unités de rayon. \end{minipage}
	\item \itempoint{1}\begin{minipage}[t]{\linewidth} Le film a commencé à $20$ h $30$. Il s'est terminé à $22$ h $25$.\\ Combien de minutes a-t-il duré ? \end{minipage}
	\item \itempoint{1}\begin{minipage}[t]{\linewidth} En $24$ minutes, un manège fait $27$ tours.\\En $8$ minutes il fait \ldots tours. \end{minipage}
\end{enumerate}
\end{multicols}



\exocorrection


    \begin{multicols}{2}

\begin{enumerate}[itemsep=1em, label=\arabic*)]
\item \begin{minipage}[t]{\linewidth}$22+13=35$\end{minipage}
\item \begin{minipage}[t]{\linewidth}$65-32=33$\end{minipage}
\item \begin{minipage}[t]{\linewidth}$17+25=42$\end{minipage}
\item \begin{minipage}[t]{\linewidth}$83-25=58$\end{minipage}
\item \begin{minipage}[t]{\linewidth}$3\times 1\,000 + 6\times 10 + 5\times 100 =3\,560$\end{minipage}
\item \begin{minipage}[t]{\linewidth}$2{,}2+3=5{,}2$\end{minipage}
\item \begin{minipage}[t]{\linewidth}$1{,}4+1{,}16=2{,}56$\end{minipage}
\item \begin{minipage}[t]{\linewidth}$5{,}63-2{,}2=3{,}43$\end{minipage}
\item \begin{minipage}[t]{\linewidth}$6{,}1-4{,}5=1{,}6$\end{minipage}
\item \begin{minipage}[t]{\linewidth}L'âge de Joachim est : $18 \div 2=9$ ans.\end{minipage}
\item \begin{minipage}[t]{\linewidth}La différence d'âge entre Léa et sa sœur est : $17-5=12$ ans.\end{minipage}
\item \begin{minipage}[t]{\linewidth}Joachim a couru : $2 \times 10=20$ minutes.\end{minipage}
\item \begin{minipage}[t]{\linewidth}${\color[HTML]{f15929}\boldsymbol{5{,}2}} - 3=2{,}2$\end{minipage}
\item \begin{minipage}[t]{\linewidth}$5 \times 6=30$\end{minipage}
\item \begin{minipage}[t]{\linewidth}$6 \times 4=24$\end{minipage}
\item \begin{minipage}[t]{\linewidth}$7{,}2-4{,}1=3{,}1$\end{minipage}
\item \begin{minipage}[t]{\linewidth}${\color[HTML]{f15929}\boldsymbol{3{,}5}} + 2{,}4=5{,}9$\end{minipage}
\item \begin{minipage}[t]{\linewidth}$7 \times 8=56$\end{minipage}
\item \begin{minipage}[t]{\linewidth}Le périmètre mesure : $3 \times 5{,}3$ cm $=15{,}9$ cm.\end{minipage}
\item \begin{minipage}[t]{\linewidth}${\color[HTML]{f15929}\boldsymbol{9{,}5}} \times 20=190$\end{minipage}
\item \begin{minipage}[t]{\linewidth}$12$ kg de fraises coûtent : $22{,}5 \times 4 = 90$\,\euro{}.\end{minipage}
\item \begin{minipage}[t]{\linewidth}${\color[HTML]{f15929}\boldsymbol{10}} \times 4=40$\end{minipage}
\item \begin{minipage}[t]{\linewidth}Le diamètre est le double du rayon : $2 \times 60 = 120$\end{minipage}
\item \begin{minipage}[t]{\linewidth}Le film a duré $1$ h $55$ min soit $115$ minutes.\end{minipage}
\item \begin{minipage}[t]{\linewidth}En $3$ fois moins de temps, ce manège fait $3$ fois moins de tours, soit : $27$ tours $\div 3=9$ tours.\end{minipage}
\end{enumerate}
\end{multicols}



\end{EXO}

\newpage

\rdexocorrection{0}